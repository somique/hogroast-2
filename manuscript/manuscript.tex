\documentclass[11pt]{article}
\usepackage{graphicx}
\usepackage{amsmath, amssymb}
\usepackage{hyperref}
\usepackage{booktabs}
\usepackage{array}
\usepackage{tabularx}
\usepackage{longtable}
\usepackage{chngpage}
\usepackage{geometry}
\usepackage{float}
\usepackage{caption}
\usepackage{placeins}
\usepackage[toc,page]{appendix}
\usepackage[backend=biber,style=numeric, sorting=none]{biblatex} % CHANGE THIS TO NUMBERS CITATIONS
\addbibresource{HOGROAST.bib}

\graphicspath{{figures/}{figure/raw_figs/}}

\title{Molecular Dynamics Informed Prediction of Antibody Thermostability}
\author{Hew Phipps \\[0.5em]
 \small Department of Statistics, The University of Oxford \\[0.1em]
 \small Supervised by Prof. Charlotte Deane, Santiago Villalba and Dr Matthew Raybould
}
\date{\today}

\begin{document}
 \maketitle
 \newpage

\section{Introduction}
Antibodies are large Y-shaped proteins constituting the adaptive
immune system's primary effectors thought to bind with high specificity 
to their targets. Typical monoclonal antibodies (mAbs) are made up 
of two pairings of heavy-light chain heterodimers where the heavy 
chains join at the stem (Fc region) via disulfide bridges. The
arms (Fab region) are split into conserved and variable sections.
The variable Fv region contains the binding interface which is 
primarily made up of three variable extended loop regions on the 
end of each chain called the CDR loops. As the Fv region determines 
binding affinity it is often the primary target in antibody design.

\paragraph{}
Historically target binding affinity has been the focal design 
objective in antibody development, mostly because before modern 
\textit{in silico} methods therapeutic design was best 
achieved by rational mutation of, and or chimeric recombination of 
Fv regions onto reliable IgG1 Immunoglobulins, the dominant 
class of antibodies in humans. Such ``CDR grafting'' has resulted in 
a multitude of successful biologic drugs, like  adalimumab, as it 
helps maintain the clinical viability of antibodies by optimising 
their ``humanness'' and other antibody attributes colloquially known 
as developability~\cite{mcconnellGeneralApproachAntibody2014}. 

\paragraph{}
Developability traits remain a major bottleneck in drug development 
pipelines. These traits often have a complex relationship with binding 
affinity~\cite{wangEarlyDeterminationPotential2024}. Computational 
prediction of developability traits therefore makes for an appealing 
solution to reducing developmental costs and timelines by providing 
the opportunity to vet drug candidates before reaching clinical 
stages where such traits could preclude advancement~\cite{jainIdentifyingDevelopabilityRisks}.
One such developability trait is thermostability. Good thermostability is
vital for manufacture, storage and shelf life~\cite{kuzmanLongtermStabilityPredictions2021}, 
reducing aggregation propensity~\cite{mcconnellGeneralApproachAntibody2014}
whilst helping prevent misfolding, and even affecting binding potency 
~\cite{maAntibodyStabilityKey2020}.

\paragraph{}
Since AlphaFold2~\cite{jumperHighlyAccurateProtein2021}, sequence and structure based deep learning models
have proven powerful predictors of protein structure. Antibody specific
models such as ImmuneBuilder~\cite{abanadesImmuneBuilderDeepLearningModels2023}, and 
generative models conditioned on binding targets~\cite{watsonNovoDesignProtein2023} perform 
well \textit{in silico} highlighting the power of machine 
learning in drug design~\cite{vecchiettiArtificialIntelligencedrivenComputational2025}. 
Indeed, the first Machine Learning (ML) aided antibody therapeutic 
to reach clinical use, Bimekizumab, shows superiority over 
previous state of the art treatments for plaque psoriasis, adalimimab 
and secukinumab~\cite{adamsBimekizumabNovelHumanized2020}, and AU-007 
represents perhaps the first antibody therapeutic designed using 
artificial intelligence in clinical trials~\cite{frentzasPhase122023}. 
Deep learning language models in particular have been frequently applied 
to the problem of thermostability prediction~\cite{pudziuvelyteTemStaProProteinThermostability2024,rodellaTemBERTureAdvancingProtein2024}
and seem to perform well at producing antibody variants with increased stability
~\cite{hutchinsonEnhancementAntibodyThermostability2024}. Deep 
learning however benefits significantly from large volumes of high 
quality data, which is readily available for sequence based learning
(almost 2 million unique paired antibody sequences and over 2 billion
unpaired sequences~\cite{olsenObservedAntibodySpace2022}) and structure
prediction (over 10,000 resolved antibody structures in the PDB 
~\cite{dunbarSAbDabStructuralAntibody2014}) but less so for 
developability (on the scale of 100s) due to these traits' 
multiparameter nature and the costs of experimental methods. 

\paragraph{}
Applying sequence or structure based deep learning methods to the task of 
thermostability prediction has so far resulted in inconsistent performance
with particularly poor results on out of training family distribution 
datasets~\cite{chungyounFLAbBenchmarkingDeep2024}, hinting at poor 
generalisability. Recent work suggests $\Delta \Delta \text{G}$ 
antibody-antigen binding prediction could require data on the order 
of $10^4$ suggesting similar numbers will be required for developability 
traits~\cite{hummerInvestigatingVolumeDiversity2025}. To further 
convolute the problem of training data, there are several 
experimental metrics for thermostability, as summarised in 
Table~\ref{tab:thermostability_metrics}.

\begin{table}
\caption{Experimental Thermostability Metrics}
\label{tab:thermostability_metrics}
\centering
\begin{tabularx}{\textwidth}{X X X}
\toprule
\textbf{Metric} & \textbf{Description} & \textbf{Experimental Method} \\
\midrule
Tm1 (Melting temperature) & Temperature at which 50\% of the protein is unfolded & Differential Scanning Fluorometry (DSF) \\
\midrule
Tm2 (Second Melting temperature) & Second melting curve transition state & DSF melt curve 2nd transition \\
\midrule
Ton (Onset temperature) & Temperature at which unfolding begins & DSF inflection point of melt curve \\
\midrule 
Tagg (Onset of Aggregation) & Temperature at which aggregation begins & Dynamic Light Scattering (DLS) \\
\bottomrule
\end{tabularx}
\end{table}

\paragraph{}
Experimental conditions such as the protein:dye ratio, 
temperature ramp rate, buffer, salts and pH, down to variations 
within the PCR instruments used, can contribute noise to 
data~\cite{boivinOptimizationProteinPurification2013,gooranFluorescenceBasedProteinStability2024,joshiApplicationNanoDifferential2020}. 
Even the method of calculating Tm from a melt curve can vary from 
taking the peak of the negative derivative, applying a two state
model or sigmoidal fit, or just the median between the baseline and 
the peak of the curve~\cite{baiIsothermalAnalysisThermoFluor2019,wuDSFworldFlexiblePrecise2024}.

\paragraph{}
This highlights a larger issue in that thermostability is a result 
of not just the chemical composition of the molecule but also the 
surrounding solvent~\cite{kuzmanLongtermStabilityPredictions2021}
and pH~\cite{fukadaLongTermStabilityReversible2018},
as shown by a 96 condition analysis of Trasuzumab 
Tm~\footnote{https://www.proteinstable.com/assets/files/pdf/Formulation-screen-of-Trastuzumab-using-the-SUPR-DSF.pdf}. 
Such external factors are difficult to incorporate into the training 
regimes of common deep learning models. Ideally, predictive models 
will be capable of generalising beyond environmental context, which 
is why many approaches focus on intrinsic chemical traits known to 
contribute to thermostability such as the number of internal contacts, 
the overall Lennard Jones (LJ) potential and the presence of salt bridges 
and disulfide bonds~\cite{peccatiAccuratePredictionEnzyme2023,zhouProCeSaContrastEnhancedStructureAware2025}. 
Aromatic ring stacking at the interface between the constant and variable 
regions has been shown to be a large contributor to the stability of 
adalimumab~\cite{yoshikawaAnalysisThermostabilitySeven2023} and the 
introduction of disulfide bonds increased denaturation temperature by 
$6.5^\circ \text{C}$~\cite{yoshikawaStabilizationAdalimumabFab2024}.

\paragraph{}
Deep learning methods may therefore benefit from added physics-based 
information. Indeed, ThermML shows how just providing Rosetta energy
features can improve thermostability classification of single chain
Fv region variants~\cite{harmalkarGeneralizablePredictionAntibody}.
Physics informed approaches like PIDiff~\cite{choiPIDiffPhysicsInformed2024b} 
have shown how incorporating the LJ of the antigen-antibody binding 
interface can significantly improve the binding affinity of diffusion 
model-generated antibodies. Such a ``physics informed'' approach could 
be of great benefit to stability prediction by reducing the reliance 
on data whilst suffering less from the likes of evolutionary bias
seen in sequence based deep learning models. For example, Chungyoun et al.~\cite{chungyounFLAbBenchmarkingDeep2024}
show how sequence based deep learning models have poor performance
predicting thermostability for antibodies out of the training set 
families and assign higher fitness to the wild type antibody 
sequence despite achieving lower thermostability than select mutants,
whilst a Rosetta energy-based model avoids this pitfall.

\paragraph{}
Proteins are dynamic and exist across a multitude of conformations responsive 
to their environment which are not easily captured by classic structure prediction 
models trained on solved crystal structures~\cite{cuiStaticStructuresProtein2025}. 
Classical statistical mechanics enables us to estimate molecule 
stability from ensembles of conformations with states more commonly 
represented being energetically more stable. Thus, equilibrium ensemble 
approaches can learn the energy landscape of a protein~\cite{lewisScalableEmulationProtein2024a},
potentially enabling thermostability prediction with higher accuracy.

\paragraph{}
MD provides a solution to the above within the limit of 
computational efficiency. The ergodic hypothesis states that 
conformational statistics captured from simulations over sufficient 
time approximate large ensembles~\cite{grossfieldQuantifyingUncertaintySampling2009}. 
As MD is an atomistic level simulation driven by Newtonian mechanics 
it also serves as a plausible ground truth for molecular physics. 
So, training a model on MD trajectories offers an opportunity for 
physics informed learning that can capture ensemble statistics. 
Such an MD informed approach for antibody thermostability prediction 
was recently applied to nanobodies~\cite{bekkerThermalStabilitySingledomain2019a} 
as well as mAb Fv regions by Merck~\cite{rollinsAbMeltLearningAntibody2024b},
who simulated 25 internal IgG1 Fv structures for 100ns at 300K, 350K and 
400K and used a number of descriptor features from the MD data to predict Tm, 
Tagg and Ton with high accuracy, a method they called AbMelt.

\paragraph{}
In this work we take inspiration from AbMelt's approach, applying 
it to a public antibody thermostability dataset of 137 therapeutic 
mAbs standardised on human IgG1 isotypes in Hepes-buffered 
saline~\cite{jainBiophysicalPropertiesClinicalstage2017a}.
We present HOGROAST, a robust and modular refactor of AbMelt for
antibody Tm prediction from GROMACS MD simulations. With HOGROAST 
we also implement antibody Fv region angular dynamics analysis using 
ABangle~\cite{dunbarABangleCharacterisingVH2013}
and provide an additional simulation protocol for coarse graining 
with Martini which enables a 7x improvement in computation time 
allowing for simulation at the microsecond level where larger 
conformational dynamics begin to arise~\cite{lindorff-larsenStructureDynamicsUnfolded2012}. 
We show how these new features also contribute to aggregation 
propensity prediction using experimental results 
from the Jain et al.~\cite{jainBiophysicalPropertiesClinicalstage2017a}
set and discuss why our results are significantly different to
AbMelt's, alongside the potential for further work.


\newpage{}

\section{Methods}

\subsection{Training Test Split}
\label{sec:Training Test Split}
A total of 137 antibody Fv region sequences and associated 
developability assay data was retrieved from Jain et al.~\cite{jainBiophysicalPropertiesClinicalstage2017a}
by csv file. Antibodys were split into train and test sets by 
clustering by Levenshtein distance sequence similarity with the 
agglomerative clustering algorithm. The train-test split was 
produced by taking the combination of clusters that came closest 
to a 70--30\% ratio in sequence counts without splitting 
clusters between sets. The test set was held out from
all analyses and training until model testing. For cross validation 
(CV), the training set was split into 3 folds 
to give 3 CV sets taking a Leave One Out strategy (LOO). Folds were
produced by clustering the train set alone by sequence similarity as 
before and grouping clusters into folds without splitting clusters 
across folds to achieve as similar antibody counts for each fold
as possible.

\subsection{Internal Contacts}
\label{sec:internal_contacts}
As in AbMelt, we define internal contacts as the number of 
intramolecular residue pairs within a distance of 3.5\AA. To 
calculate this for static pdb structures, we use Biopython's
PDB parser tools~\cite{cockBiopythonFreelyAvailable2009} and 
iterate through all possible atom pairs, tallying a residue pair 
once if at least one atom pair is within 3.5\AA. We used 
MDAnalysis~\cite{michaud-agrawalMDAnalysisToolkitAnalysis2011}
to load trajectories and its FastNS neighbour search function to 
count the number of contacts for each frame.

\subsection{Correlation Analysis}
\label{sec:Correlation Analysis}
All correlations throughout this work refer to Pearson's correlation
coefficient ($r_p$), which denotes the strength and correlation direction
(slope of the gradient) of the optimal linear relationship between
two variables (Equation~\ref{eq:pearson}).

\begin{equation}
r_p = \frac{\text{cov}(X, Y)}{\sigma_X \sigma_Y}
\label{eq:pearson}
\end{equation}
Where $X$ and $Y$ are sets of observations and $\sigma_X$ and 
$\sigma_Y$ are their standard deviations. Within this text we 
refer to the 11 Jain et al. assay metrics for each antibody as 
"fitness features" (Table~\ref{tab:jain_metrics}) corresponding
with $Y$ whilst the MD descriptor features for each antibody 
correspond to $X$.

\paragraph{}
To compare different datasets of correlations between antibody MD
features and fitness features such as Tm we use a number of 
matrix comparison statistics. For two datasets of Pearson's $r_p$ 
correlation coefficients as matrices $A \in \mathbb{R}^{n \times m}$ 
and $B \in \mathbb{R}^{n \times m}$ where each row $a_i$ and $b_i$ 
corresponds to the vector of correlations for sample $i$ of $n$ 
across $m$ features, we compute:

\begin{enumerate}
  \item The mean $\mu_r$ of all $|r_p|$ for each dataset is computed as:
      \begin{equation}
      \mu_r = \frac{1}{n \cdot m} \sum_{i=1}^{n} \sum_{j=1}^{m} |r_p^{i,j}|
      \label{eq:mean_rp}
      \end{equation}
  \item A paired t-test $t$ on Fisher's z-transformed $r_p$ ($z$) to determine if the means of the two datasets are significantly different is computed as:
      \begin{align}
      \bar{d} &= \frac{1}{n} \sum_{i=1}^n (z_a^i - z_b^i) \\
      s_d &= \sqrt{\frac{1}{n-1} \sum_{i=1}^{n} \bigl( (z_{a}^{i} - z_{b}^{i}) - \bar{d} \bigr)^2 } \\
      t &= \frac{\bar{d}}{s_d / \sqrt{n}} 
      \label{eq:t_test}
      \end{align}
  \item A Wilcoxon signed-rank test $W$ on Fisher's z-transformed $r_p$, a non-parametric test to determine if the median of differences of the two datasets is significantly different from zero, is computed as:
      \begin{align}
      d_i &= z_a^i - z_b^i \\
      R_i &= \operatorname{rank}(|d_i|) \\
      W &= \sum_{i=1}^n \operatorname{sgn}(d_i) \cdot R_i 
      \label{eq:wilcoxon}
      \end{align}
   \item The difference in the Frobenius norms $\|A-B\|_F$ of two matrices of $r_p$ values was computed to gauge the difference in overall magnitude of correlations with:
      \begin{equation}
      \|A-B\|_F = \sqrt{ \sum_{i=1}^m \sum_{j=1}^n (a^{i,j} - b^{i,j})^2 }
      \label{eq:frobenius}
      \end{equation}
\end{enumerate}

Where the Fisher's z-transform, $z$, used to remove skew from the correlation matrices and approximate normality, is defined by:
\begin{equation}
z = \frac{1}{2} \ln\left( \frac{1 + r_p}{1 - r_p} \right)
\label{eq:fisher_z}
\end{equation}

\paragraph{}
To determine what $r_p$ correlations are to be considered significant
we used a one-sample t-test for significance (Equation~\ref{eq:one_sample}) 
for each $r_p$ in each dataset to get the number of significantly 
correlated features and a critical $r_p$ ($r_{crit}$) at which 
correlations less than this are rejected under the null hypothesis.

\begin{align}
t &= \frac{r_p \sqrt{n - 2}}{\sqrt{1 - (r_p)^2}} \\
r_{\text{crit}} &= \frac{t_{\text{crit}}}{\sqrt{t_{\text{crit}}^2 + df}}
\label{eq:one_sample}
\end{align}

Where the degrees of freedom $df$ is equal to $n-2$ and $t_{crit}$ 
is obtained from a two tailed test equal to the $(1-\alpha/2)$ 
percentile of the Student's t-distribution where we used an $\alpha$
of 0.05 by default. 


\subsection{Baselines}
\label{sec:Establishing Baselines}
The database TheraSAbDab~\cite{raybouldTheraSAbDabTherapeuticStructural2020} 
was used to find PDB crystal structures for antibody sequences in 
the Jain et al. dataset. On 26/07/25 all entries in the database were 
downloaded from the TheraSAbDab webserver as a csv file. Using 
Python, the data was scraped for antibodies sharing the same name 
as any in the Jain et al. dataset. Columns pertaining to antibody 
name and PDB ID for structures with 100\%, 99\% or 95-98\% sequence 
similarity were collected.

\paragraph{}
For some antibodies, several PDB entries were available each
listed with their respective heavy and light chain IDs for that PDB structure
(often non-canonically labelled). The first PDB code and chain
ID per structure was taken for each of the 100\%--99\% sequence 
identity PDBs and used to scrape structures from the RSCB PDB 
site, retrieving a PDB file for each. As the retrieved structures 
were Fab regions they were renumbered with ANARCI~\cite{dunbarANARCIAntigenReceptor2016} 
to extract Fv regions before hydrogens were added with PDBfixer. 
Internal contacts for each structure was determined based on the 
definition outlined in Section~\ref{sec:internal_contacts}. The 
resultant structure contacts per antibody were plotted against 
their respective Tm values to obtain a baseline correlation.

\paragraph{}
For benchmarks more comparable to molecular dynamics trajectory
data antibody Fv region structure ensembles were generated for each 
antibody in the Jain set with ImmuneBuilder~\cite{abanadesImmuneBuilderDeepLearningModels2023}
using the ABodyBuilder2 model. Briefly, for each
antibody the heavy and light chain sequences were fed into four 
separate models to predict 4 separate structures. The closest 
structure to the average was selected then relaxed with 
OpenMM~\cite{eastmanOpenMM8Molecular2024} to remove structural 
clashes. This was repeated 100 times for each antibody in the Jain
dataset to produce a final ensemble from which the mean and 
standard deviation of internal contacts across the ensemble was
taken and their correlations with Tm determined.

\subsection{Molecular Dynamics Simulation with GROMACS}
All molecular dynamics simulations were performed with 
GROMACS-2025.1~\cite{abrahamGROMACS20251Source2025}. 
Structures were predicted from VH and VL sequences with ImmuneBuilder's
ABodyBuilder2~\cite{abanadesImmuneBuilderDeepLearningModels2023}. 
Histidine residue protonation states were determined by predicting 
each structures' pKa using propka 3.1~\cite{olssonPROPKA3ConsistentTreatment2011,sondergaardImprovedTreatmentLigands2011}
and deprotonating if it was below the physiological pH of 7.4 at which 
all simulations were conducted. CDR loops regions were identified 
with IMGT numbering using ANARCI~\cite{dunbarANARCIAntigenReceptor2016}, 
and for solved PDB structures this was also used to identify heavy 
and light chains.

\paragraph{}
All all-atom simulations followed the same protocol set out in AbMelt~\cite{rollinsAbMeltLearningAntibody2024b}. 
In brief, all-atom simulations used the CHARMM27 (CHARMM22 $+$ CMAP) 
force field~\cite{mackerellExtendingTreatmentBackbone2004}. The 
system was solvated with ~9000 water molecules with the TIP3P water 
model~\cite{jorgensenComparisonSimplePotential1983} in rectangular
boxes of ~$5.8 \times 7.5 \times 7.3\,\mathrm{nm}^3$ and ~60 Na+ and 
Cl- ions were added to reach a salt concentration of ~$150\,\mathrm{mM}$. 
The total number of atoms reached ~30,000 per simulation. The 
LeapFrog algorithm was used with $2\,\mathrm{fs}$ timesteps. First

\paragraph{}
Simulations were carried out at each temperature of 300K, 350K, 400K
with four steps:
\begin{enumerate}
\item The system was relaxed with steepest descent energy minimization.
\item Simulation with constant particle, volume and temperature (NVT) 
canonical ensemble for $100\,\mathrm{ps}$ using the velocity rescale 
(V-rescale) thermostat with a $0.1\,\mathrm{ps}$ time constant.
\item Simulation with constant particle, pressure and temperature (NPT) 
ensemble for $100\,\mathrm{ps}$ using the Berendsen barostat and a 
$2.0\,\mathrm{ps}$ time constant to maintain isotropic pressure at 
1.0~\text{bar}.
\item Production simulation without restraints for $100\,\mathrm{ns}$
with the Nose-Hoover thermostat and Parrinello-Rahman barostat to 
maintain temperature and pressure with $2.0\,\mathrm{ps}$ and 
$1.0\,\mathrm{ps}$ time constants, respectively, whilst saving 
frame information every $10\,\mathrm{ps}$.
\end{enumerate}

\paragraph{}
Long range electrostatics used the Particle Mesh Ewald algorithm (PME)
~\cite{dardenParticleMeshEwald1993}. The Verlet cutoff scheme~\cite{pallFlexibleAlgorithmCalculating2013} 
was used to determine short range non-bonded interactions. Bond 
lengths were constrained with the LINCS algorithm~\cite{hessLINCSLinearConstraint1997} 
and water with the SHAKE algorithm~\cite{ryckaertNumericalIntegrationCartesian1977}.

\paragraph{}
Simulation trajectories were fixed to correct for periodic boundary 
jumps then the final trajectories were stripped of added water 
molecules and ions and saved as compressed .xtc files for use with 
GROMACS tools and MDAnalysis.

\subsection{Feature Extraction}
Features were extracted from .xtc and relevant topology files using 
built-in GROMACS tools with the IMGT defined group indexes and 
removing the first $20\,\mathrm{ns}$ of simulation as further equilibration 
time following AbMelt's approach~\cite{rollinsAbMeltLearningAntibody2024b}. 
The following GROMACS tools listed in Table~\ref{tab:gmx_commands} were used.

{\tiny
\begin{table}
   \caption{GROMACS Tools Used For Feature Extraction}
   \label{tab:gmx_commands}
   \newcolumntype{L}[1]{>{\raggedright\arraybackslash}p{#1}}  % Left-aligned fixed width
   \newcolumntype{C}[1]{>{\centering\arraybackslash}p{#1}}   % Centered fixed width
   \newcolumntype{R}[1]{>{\raggedleft\arraybackslash}p{#1}}  % Right-aligned fixed width
   \centering
      \begin{tabularx}{\linewidth}{L{2cm} L{2.9cm} L{3.1cm} C{1.5cm} C{1.5cm}}
      \toprule
      \textbf{GROMACS Command} & \textbf{Feature} & \textbf{Region} & \textbf{Data Type} & \textbf{Units} \\
      \midrule
      \texttt{gmx sasa} & Solvent Accessible Surface Area & Fv, H, L, CDRS, CDRH1, CDRH2, CDRH3, CDRL1, CDRL2, CDRL3 & time series & $\mathrm{nm}^2$ \\
       \texttt{gmx mindist} & Internal contacts count &   Fv, H, L, CDRS, CDRH1, CDRH2, CDRH3, CDRL1, CDRL2, CDRL3 & time series & count \\
        \texttt{gmx hbond} & Hydrogen bond count & Fv, H, L & time series & count \\
         \texttt{gmx rms} & Root Mean Square deviation &  Fv, H, L, CDRS, CDRH1, CDRH2, CDRH3, CDRL1, CDRL2, CDRL3 & time series & $\mathrm{nm}$ \\
          \texttt{gmx gyrate} & Radius of gyration &  Fv, H, L, CDRS, CDRH1, CDRH2, CDRH3, CDRL1, CDRL2, CDRL3 & time series & $\mathrm{nm}$ \\
           \texttt{gmx dipole} & Dipole moment & Fv & time series & Debye \\
            \texttt{gmx energy} & Energies & Fv & time series & various \\
             \texttt{gmx rmsf} & Root Mean Square Fluctuation &  Fv, H, L, CDRS, CDRH1, CDRH2, CDRH3, CDRL1, CDRL2, CDRL3 & per atom & $\mathrm{nm}$ \\
              \texttt{gmx covar} & Covariance matrix of atomic fluctuations & Fv & NA & NA \\
               \texttt{gmx anaeig} & Eigen vector analysis of covariance matrix & Fv & Eigen values & $\mathrm{nm}/S^2/\mathrm{N}$ \\
                \texttt{gmx potential} & Electrostatic potentials, charges and field energy across atom groups &  Fv, H, L, CDRS, CDRH1, CDRH2, CDRH3, CDRL1, CDRL2, CDRL3 & z axis average coordinate & various \\
      \bottomrule
   \end{tabularx}
\end{table}
}

\paragraph{}
Additionally, a number of extra features were computed from the 
above, namely:
\begin{enumerate}
\item Partition-SASA was calculated as the SASA per residue using the
Shrake-Rupley algorithm~\cite{shrakeEnvironmentExposureSolvent1973} 
with MDAnalysis then selecting core and surface residues by the 20 
residues with the least and most SASA respectively to get core and 
surface partition-wise SASA. 
\item N-H backbone vector order parameter $S^2$ was calculated as in 
AbMelt with a block size of $10\,\mathrm{ns}$ and excluding prolines~\cite{rollinsAbMeltLearningAntibody2024b}.
\item The Lambda metric is calculated as the temperature dependence of 
S\textsuperscript{2} by the gradient and $r_p$ fit of $\ln(1-S)$ vs 
$\ln(T)$, as described in AbMelt~\cite{rollinsAbMeltLearningAntibody2024b}
\end{enumerate}

\paragraph{}
All feature data was collected for each antibody into a single 
dataset of both means and stdevs (summarised in Table~\ref{tab:all_features})
of each feature at each temperature which was used for all 
subsequent correlation analyses. The data was further processed for 
model feature selection and training by filtering out erronous 
results and replacing any missing values present by 
imputation with means to preserve the underlying statistics.

\subsection{ABangle}
ABangle~\cite{dunbarABangleCharacterisingVH2013} is a tool used for 
the calculation of antibody Fv region inter-chain angles (Figure~\ref{fig:abangle1}). 
Specifically, by setting a centroid vector between the structural 
mass centroids of the heavy and light chain it calculates 5 angles: 
HL, HC1, LC1, HC2, LC2 (Table~\ref{tab:abangle_angles}).

\begin{table}
\caption{ABangle Fv Region Angles}
\newcolumntype{L}[1]{>{\raggedright\arraybackslash}p{#1}}
\newcolumntype{R}[1]{>{\raggedleft\arraybackslash}p{#1}} 
\label{tab:abangle_angles}
\centering
\begin{tabularx}{\textwidth}{L{2cm} L{10cm}}
\toprule
\textbf{Angle} & \textbf{Definition} \\
\midrule
HL & Angle between heavy and light chain vectors HC1 and LC1 along the centroid plane bissecting the two chains \\
 HC1 & Angle between the HC1 and centroid vectors \\
  LC1 & Angle between the LC1 and centroid vectors \\
   HC2 & Angle between the HC2 and centroid vectors \\
    LC2 & Angle between the LC2 and centroid vectors \\
\bottomrule
\end{tabularx}
\end{table}

\paragraph{}
To complement the angle analysis we also describe 6 distances based 
on four points, Hp1 and Lp1 which are each set $10\text{\AA}$ from the 
centroid point of each chain in the direction of vectors HC1 and
LC1 respectively, and Hp2 and Lp2 similarly for vectors HC2 and
LC2, respectively (Table~\ref{tab:abangle_distances}).

\begin{table}
\caption{ABangle Fv Region Distances}
\newcolumntype{L}[1]{>{\raggedright\arraybackslash}p{#1}}
\newcolumntype{R}[1]{>{\raggedleft\arraybackslash}p{#1}} 
\label{tab:abangle_distances}
\centering
\begin{tabularx}{\textwidth}{L{2cm} L{10cm}}
\toprule
\textbf{Distance} & \textbf{Definition} \\
\midrule
DC & Length of the centroid vector \\
 PP & Distance between points Hp1 and Lp1 \\
  H1 & Distance between the point Lp1 and the centroid of chain H \\
   L1 & Distance between the point Hp1 and the centroid of chain L \\
    H2 & Distance between the point Lp2 and the centroid of chain H \\
     L2 & Distance between the point Hp2 and the centroid of chain L \\
\bottomrule
\end{tabularx}
\end{table}

\paragraph{}
Analysis was performed with MDAanalysis on downsampled trajectories
for computational efficiency where every 20th frame was converted to 
a temporary pdb file for processing with ABangle. For details on how 
ABangle defines the centroid vector and angles see~\cite{dunbarABangleCharacterisingVH2013}.


\subsection{Martini Coarse Graining}
\label{sec:martini}
Coarse graining improves computational performance by orders of 
magnitude, allowing for significantly extended simulation times with 
minimal loss of accuracy~\cite{souzaMartini3General2021}. The 
Martini3~\cite{souzaMartini3General2021} coarse grain force field 
splits residues into a backbone bead comprising the backbone atoms 
and a variable number of beads capturing side chains. 

\paragraph{}
To coarse grain an all-atom PDB, the Martinise tool from the package
Vermouth~\cite{kroonMartinize2VermouthUnified2025} was used with elastic 
network (harmonic spring) parameters of a $700\,\mathrm{kJ}/\mathrm{mol}/
\mathrm{nm}^2$ spring constant, a lower distance cutoff of $0.5\,\mathrm{nm}$ 
and upper distance cutoff of $0.5\,\mathrm{nm}$. Coarse 
graining performs best on energetically stable structures which can 
be obtained from a prior molecular dynamics simulation or with an 
ensemble of structures where the most abundant conformation is 
deemed most stable. To avoid further MD adding to computation time, 
we opted for an ensemble strategy using ImmuneBuilder to generate 
100 unique structures for each Fv region. A representative structure 
was obtained by capturing structural variation within 3D space across 
the ensemble with Principle Component Analysis (PCA) and clustering 
then taking the medoid structure of the most populated cluster as a 
loose approximation of the most stable conformation.

\paragraph{}
After coarse graining the representative structure the system was 
solvated with Insane~\cite{wassenaarComputationalLipidomicsInsane2015}. 
We used a periodic box size of 7 units, a salt concentration of 0.15 
mol/L, and solvated the system with 100 water molecules. As with the 
all-atom protocol, the system underwent energy minimisation with 
steepest descent before being equilibrated with NVT and NPT 
canonical ensembles this time for $200\,\mathrm{ps}$, followed by 
production in NPT.

\paragraph{}
Unlike the all-atom configuration which used Nose-Hoover,
we use the V-rescale thermostat as we found it more robust for 
coarse grain systems which don't reproduce exact atomic dynamics. 
We used a timestep of $10\,\mathrm{fs}$, saving frame information 
every $10\,\mathrm{ps}$. As with all-atom, the LINCS algorithm and 
Verlet cutoff schemes were used for bond constraints and neighbour 
searching respectively. The C-rescale barostat with isotropic 
pressure coupling was used for pressure scaling.

\paragraph{}
Since coarse graining removes atomistic resolution a number 
of analyses are unavailable, such as hydrogen bond analysis which 
relies on proton acceptor pairs. Thus, to produce the same features 
analysis the final processed trajectories were 
converted back to all-atom (finegraining) with the cg2all package 
~\cite{heoOneParticleResidue2023} and subjected to the same 
analysis as the all-atom protocol. Due to the nature of energy 
analysis being performed on the coarse grained trajectory not the 
cg2all trajectory the following features had to be excluded from 
martini feature extraction: 
lj14, ljSR, coloumb14, columbSR, kinetic, potential, and enthalpy
(Table~\ref{tab:all_features}). Further, due to issues with 
cg2all sometimes producing trajectory frames with overlapping atoms
interfering with MDAnalysis' sasa analysis tool the partition-sasa 
feature was also excluded.

\paragraph{}
RMSD analysis of cg2all trajectories was performed after aligning the
cg2all trajectory frames and original all-atom frames to the first 
frame in the all original atom trajectory as reference. RMSD was 
calculated using MDAnalysis's rmsd function on backbone atoms.

\paragraph{}
To compare feature values between all-atom and cg2all simulations 
all missing values were filtered out, and any features not shared 
between the coarse grain analysis and all-atom were removed. The 
difference matrix $\mathbf{F}_{diff} \in \mathbb{R}^{n \times m}$ 
between values for every feature ($m$) for every antibody ($n$) was 
computed with:
\begin{equation}
\mathbf{F}_{diff} = \mathbf{F}_{aa} - \mathbf{F}_{cg2all}
\end{equation}
Where $\mathbf{F}_{aa} \in \mathbb{R}^{n \times m}$ and 
$\mathbf{F}_{cg2all} \in \mathbb{R}^{n \times m}$ refer to the 
matrices of antibody (rows) and feature (cols) values for the all
atom and coarse grained (finegrained) simulations respectively.

\paragraph{}
To gauge the similarity between the two feature datasets the 
RV coefficient, which captures shared variance and agreement across 
samples between two datasets, was computed to give a value between 0
and 1 where 1 indicates the strongest possible overall structural 
similarity between the two datasets, as follows:

\begin{equation}
\mathrm{RV}(\mathbf{A}, \mathbf{B}) = \frac{\langle \mathbf{A}\mathbf{A}^\top, \mathbf{B}\mathbf{B}^\top \rangle_F}{\|\mathbf{A}\mathbf{A}^\top\|_F \cdot \|\mathbf{B}\mathbf{B}^\top\|_F}
\label{eq:rv_coefficient}
\end{equation}

Where $\mathbf{A}$ and $\mathbf{B}$ refer to the mean centered 
$\mathbf{F}_{aa}$ and $\mathbf{F}_{cg2all}$ per antibody feature
matrices for the all-atom and coarse grained sets respectively. 
Mean centering was performed by subtracting the row mean 
of each row in the matrix from every value in the respective row
given rows are samples (antibodies) and columns features.

\subsection{Machine Learning on Descriptor Features}
Machine learning on MD features was performed as follows. First, highly 
inter-correlated features (Pearson $r_p$ coefficient $>$ 0.95) were 
removed. Recursive followed by exhaustive feature selection with 
CV was performed using the 3-fold train validation splits and the 
scikit-learn~\cite{pedregosaScikitlearnMachineLearning2011} 
package's RFECV and EFSCV modules with its Random Forest Regressor 
as the model fit metric whilst optimising hyperparameters with 
Optuna~\cite{akibaOptunaNextgenerationHyperparameter2019} using 
50 trials. This was performed separately for each developability 
feature from the Jain et al. set.

\paragraph{}
For each of the developability fitness features, six 
models were trained from their selected MD descriptor feature 
subsets (Table~\ref{tab:models}). During training the same CV split 
as feature selection was used to optimise hyperparameters and select 
the optimal performing model before testing.

\begin{table}
\caption{Table of models used in training on each developability assay metric (fitness feature) selected features subset}
\label{tab:models}
\centering
\begin{tabularx}{\textwidth}{X X X}
\toprule
\textbf{Model} & \textbf{Description} & \textbf{Package} \\
\midrule
Elastic Net & Regression with L1 and L2 regression for high dimensional data where the number of predictors is $>$ the number of samples & Scikit-learn \\
 K Nearest Neighbours (KNN) & Non-parametric algorithm that takes the average of \boldmath{K} closest neighbours as the regression prediction & Scikit-learn \\
  Linear regression & Models the optimal linear relationship between predictors and an outcome  & Scikit-learn \\
   Random forest & Ensemble of decision trees where each tree is trained on a random subset of the data then takes the average of tree predictions as regression output & Scikit-learn \\
    Support Vector Machine (SVM) & Finds a function that approximates outcomes within a margin of tolerance & Scikit-learn \\
     XG Boost & Sequential tree building with L1 and L2 regularisation minimising a loss function that corrects errors from previous trees to find an optimal objective function & Scikit-learn \\
\bottomrule
\end{tabularx}
\end{table}

\paragraph{}
Each trained model for each developability fitness feature was 
tested on the holdout test split. Accuracy was computed with
three metrics between the predicted value and the ground truth 
from the Jain et al. set: The coefficient of determination ($r^2$), 
Pearson's coefficient ($r_p$) and Spearman's coefficient ($\rho$)
as below:

\begin{equation}
r^2 = (r_p)^2
\label{eq:r_squared}
\end{equation}

\begin{equation}
\rho = 1 - \frac{6 \sum_{i=1}^{n} d_i^2}{n(n^2 - 1)}
\label{eq:spearman}
\end{equation}

\newpage

\clearpage
\printbibliography[heading=bibintoc]

\newpage

\renewcommand{\appendixname}{Supplementary Information}
\renewcommand{\thesection}{SI~\arabic{section}}
\renewcommand{\thesubsection}{SI~\arabic{section}.\arabic{subsection}}
\renewcommand{\thetable}{S\arabic{table}}
\renewcommand{\thefigure}{S\arabic{figure}}

\section*{Supplementary Information}
\addcontentsline{toc}{section}{Supplementary Information}
\setcounter{subsection}{0}
\setcounter{table}{0} 
\setcounter{figure}{0} 

\setlength{\LTcapwidth}{\textwidth}
\begin{longtable}{p{0.35\textwidth} p{0.25\textwidth} p{0.25\textwidth}}
\caption{PDB structures for antibodies from the Jain et al. set retrieved from RCSB PDB}\label{tab:supp-S1}%
\\

\toprule
\textbf{Name} & \textbf{PDB ID} & \textbf{Chain IDs} \\
\midrule
\endfirsthead

\toprule
\textbf{Name} & \textbf{PDB ID} & \textbf{Chain IDs} \\
\midrule
\endhead

\midrule
\multicolumn{3}{r}{\textit{Continued on next page}} \\
\endfoot

\bottomrule
\endlastfoot

Adalimumab & 6cr1 & HL \\
Alemtuzumab & 1bey & HL \\
Anifrolumab & 4qxg & HL \\
Atezolizumab & 5x8l & FK \\
Basiliximab & 1mim & HL \\
Belimumab & 5y9k & HL \\
Bevacizumab & 1bj1 & HL \\
Bimagrumab & 5nhr & HL \\
Bococizumab & 3sqo & HL \\
Briakinumab & 5n2k & LK \\
Certolizumab & 5wuv & HL \\
Cetuximab & 6au5 & DC \\
Crenezumab & 5vzx & HL \\
Dacetuzumab & 8yx9 & HL \\
Daclizumab & 3nfs & HL \\
Daratumumab & 7dun & HL \\
Drozitumab & 4od2 & BA \\
Dupilumab & 6wgb & AB \\
Eculizumab & 5i5k & HL \\
Efalizumab & 3eo9 & HL \\
Gantenerumab & 5csz & HL \\
Gemtuzumab & 8w70 & BA \\
Gevokizumab & 4g6k & HL \\
Golimumab & 5yoy & RO \\
Guselkumab & 4m6m & HL \\
Ibalizumab & 3o2d & HL \\
Infliximab & 5vh3 & HL \\
Ipilimumab & 5tru & HL \\
Ixekizumab & 6nov & AB \\
Lampalizumab & 4d9q & HL \\
Lebrikizumab & 4i77 & HL \\
Lirilumab & 8tui & HL \\
Matuzumab & 3c08 & HL \\
Motavizumab & 3qwo & AB \\
Muromonab & 1sy6 & HL \\
Necitumumab & 6b3s & JK \\
Nivolumab & 5ggq & HL \\
Obinutuzumab & 3pp3 & HL \\
Ofatumumab & 3giz & HL \\
Olokizumab & 4cni & HL \\
Omalizumab & 4x7s & HL \\
Onartuzumab & 4k3j & HL \\
Panitumumab & 5sx5 & HL \\
Pembrolizumab & 5dk3 & GF \\
Pertuzumab & 1l7i & HL \\
Ponezumab & 3u0t & BA \\
Radretumab & 7ah1 & AA \\
Ramucirumab & 3s34 & HL \\
Ranibizumab & 1cz8 & HL \\
Rituximab & 2osl & HL \\
Sarilumab & 8iow & HL \\
Secukinumab & 6wio & AB \\
Sifalimumab & 4ypg & HL \\
Tanezumab & 4edw & HL \\
Tocilizumab & 8j6f & HL \\
Tralokinumab & 5l6y & HL \\
Trastuzumab & 6bi0 & HL \\
Tremelimumab & 5ggu & HL \\
Urelumab & 6mhr & AB \\
Ustekinumab & 3hmw & HL \\
Canakinumab & 4g5z & HL \\
Carlumab & 4dn3 & HL \\
Epratuzumab & 5vkk & AB \\
Fresolimumab & 3eo0 & DC \\
Fulranumab & 4m6o & HL \\
Lumiliximab & 3fzu & HL \\
Natalizumab & 4irz & HL \\
Nimotuzumab & 3gkw & HL \\
Palivizumab & 2hwz & HL \\
Pinatuzumab & 6and & HL \\

\end{longtable}


\newpage 

\begin{table}
\caption{Jain et al. Antibody developability metrics}
\label{tab:jain_metrics}
\centering
\begin{tabular}{p{0.35\textwidth} p{0.55\textwidth}}
\toprule
\textbf{Metric} & \textbf{Description} \\
\midrule
Cross-Interaction Chromatography (CIC) & Measures non-specificity and propensity for aggregation against polyclonal human serum antibodies \\
 PSR Binding Assay & Measures polyspecificity by flow cytometry with biotinylated human membrane proteins \\
  CSI-BLI & Measures antibody aggregation propensity with itself \\
   AC-SINS & Another method for antibody self-aggregation propensity \\
    SGAC-SINS & AC-SINS under high salt conditions \\
     Tm (DSF) & Melting temperature measurement using differential scanning fluorometry (DSF) \\ 
      SMAC & Aggregation, precipitation and solubility assessment \\ 
       AS SEC Slope & Aggregation and self-association \\ 
        HIC & Measures surface hydrophobicity of antibodies for aggregation and solubility assessment \\ 
         BVP Assay & Polyspecificity, immunogenicity and pharmocokinetics by assessing binding to Baculovirus Particles \\ 
          ELISA & Measures antibody expression levels \\ 
\bottomrule
\end{tabular}
\end{table}


\newpage

\clearpage
\begin{figure}[H]
\centering
\includegraphics[width=\textwidth]{epratuzumab_rmsd.png}
\caption{RMSD of epratuzumab all-atom simulation at 300K, 350K, and 400K for 250 ns. Each frame in the x axis represents a $10\,\mathrm{ps} time step$. Dashed red vertical lines mark the 100 ns timepoint. (A) Cumulative mean of the RMSD over the trajectory. (B) Cumulative Stdev of the RMSD over the trajectory. (C) RMSD final frame mean (top) and stdev (bottom) values for each temperature. (D) RMSD over the trajectory.}
\label{fig:epratuzumab_rmsd}
\end{figure}


\newpage

\clearpage
\begin{figure}[H]
\centering
\includegraphics[width=\textwidth]{raw_figs/epratuzumab_differences.png}
\caption{Difference between cumulative summary statistics of the internal contacts of the 100 ns (post-20 ns equilibration cuttoff) and first 80 ns (post-20 ns equilibration cuttoff) of the 250 ns epratuzumab simulations. (Left) difference between cumulative means. (Right) difference between cumulative stdevs.}
\label{fig:epratuzumab_differences}
\end{figure}


\newpage

\setlength{\LTcapwidth}{\textwidth}
\begin{longtable}{p{0.35\textwidth} p{0.55\textwidth}}
\caption{Table listing all descriptor features extracted from MD simulations in this work. The "Group Name" column groups the individual features that each examine a respective part of the group feature.}\label{tab:all_features}%
\\
\toprule
\textbf{Group Name} & \textbf{Feature} \\
\midrule
\endfirsthead

\toprule
\textbf{Group Name} & \textbf{Feature} \\
\midrule
\endhead

\midrule
\multicolumn{2}{r}{\textit{Continued on next page}} \\
\endfoot

\bottomrule
\endlastfoot

contacts & contacts mean \\
contacts & contacts std \\
sasa & sasa mean \\
sasa & sasa std \\
partition-sasa & partition-sasa mean \\
partition-sasa & partition-sasa std \\
per-res-s2 & per-res-s2 mean \\
per-res-s2 & per-res-s2 std \\
per-block-s2 & per-block-s2 mean \\
per-block-s2 & per-block-s2 std \\
angles & HLangle mean \\
angles & HLangle std \\
angles & HC1angle mean \\
angles & HC1angle std \\
angles & LC1angle mean \\
angles & LC1angle std \\
angles & HC2angle mean \\
angles & HC2angle std \\
angles & LC2angle mean \\
angles & LC2angle std \\
distances & DCdistance mean \\
distances & DCdistance std \\
distances & PPdistance mean \\
distances & PPdistance std \\
distances & H1fromCLdistance mean \\
distances & H1fromCLdistance std \\
distances & L1fromCHdistance mean \\
distances & L1fromCHdistance std \\
distances & H2fromCLdistance mean \\
distances & H2fromCLdistance std \\
distances & L2fromCHdistance mean \\
distances & L2fromCHdistance std \\
bonds & hbonds mean \\
bonds & hbonds std \\
covar & covar mean \\
covar & covar std \\
rmsd & rmsd mean \\
rmsd & rmsd std \\
gyr & Rgyr mean \\
gyr & Rgyr std \\
gyr & Rgyr Xaxis mean \\
gyr & Rgyr Xaxis std \\
gyr & Rgyr Yaxis mean \\
gyr & Rgyr Yaxis std \\
gyr & Rgyr Zaxis mean \\
gyr & Rgyr Zaxis std \\
dipole & dipole X mean \\
dipole & dipole X std \\
dipole & dipole Y mean \\
dipole & dipole Y std \\
dipole & dipole Z mean \\
dipole & dipole Z std \\
dipole & dipole T mean \\
dipole & dipole T std \\
lj14 & LJpotential 1-4 mean \\
lj14 & LJpotential 1-4 std \\
ljSR & LJpotential short-range mean \\
ljSR & LJpotential short-range std \\
coloumb14 & coulomb 1-4 mean \\
coloumb14 & coloumb 1-4 std \\
coloumbSR & coulomb short-range mean \\
coloumbSR & coloumb short-range std \\
potential & potential energy mean \\
potential & potential energy std \\
kinetic & kinetic energy mean \\
kinetic & kinetic energy std \\
energy & total energy mean \\
energy & total energy std \\
temperature & temperature mean \\
temperature & temperature std \\
pv & pressur volume mean \\
pv & pressure volume std \\
enthalpy & enthalpy mean \\
enthalpy & enthalpy std \\
charge & charge mean \\
charge & charge std \\
rmsf & rmsf mean \\
rmsf & rmsf std \\

\end{longtable}

\newpage

\clearpage
\begin{figure}[H]
\centering
\includegraphics[width=\textwidth]{raw_figs/epratuzumab_nvt_equilibration.png}
\caption{Epratuzumab Martini3 coarse grained equilibration is stable for 100 ps with the NVT canonical ensemble at each temperature. (A) Kinetic energy. (B) Potential energy. (C) Temperature.}
\label{fig:epratuzumab_nvt}
\end{figure}


\newpage

\clearpage
\begin{figure}[H]
\centering
\includegraphics[width=\textwidth]{raw_figs/epratuzumab_npt_equilibration.png}
\caption{Epratuzumab Martini3 coarse grained equilibration converges over 100 ps with the NPT canonical ensemble at each temperature. (A) Kinetic energy. (B) Potential energy. (C) Temperature.}
\label{fig:epratuzumab_npt}
\end{figure}


\end{document}